\newpage
\section{Appendix}
\subsection{Official Assignment} \label{offassignment}
\subsubsection*{\textbf{Titel}}
Misinformation Detection
\subsubsection*{\textbf{Beschreibung}} \label{offassignment_description}
Diese Bachelorarbeit zielt darauf ab, ein fortschrittliches Tool zu entwickeln, das auf Deep Learning-Techniken basiert, um Desinformationen im Internet effektiv zu erkennen und zu analysieren. Im Zentrum der Arbeit steht die Erweiterung und Verfeinerung eines bestehenden Deep Learning-Modells, um es speziell für die Herausforderungen und Dynamiken des Erkennens von falschen Informationen im Web anzupassen.

Forschungsziele:
\begin{itemize}
    \item Untersuchung der aktuellen Landschaft der Desinformation im Internet und Identifizierung spezifischer Herausforderungen und Muster.
    \item Analyse bestehender Deep Learning-Modelle und -Methoden, die für die Erkennung von Desinformationen relevant sind.
    \item Entwicklung und Anpassung eines Deep Learning-basierten Tools, das speziell auf die Erkennung von Desinformationen im Web ausgerichtet ist.
    \item Durchführung von Tests und Evaluationen, um die Wirksamkeit und Genauigkeit des Tools in verschiedenen Szenarien zu bewerten.
\end{itemize}

Ziel ist die Teilnahme an dem CheckThat-Lab 2024.

\subsubsection*{\textbf{Voraussetzungen}}
\begin{itemize}
    \item Programmierkenntnisse in Python
    \item Erste Erfahrungen im Deep Learning von Vorteil
    \item Hohe Motivation
\end{itemize}



